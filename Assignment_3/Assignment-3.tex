% Options for packages loaded elsewhere
\PassOptionsToPackage{unicode}{hyperref}
\PassOptionsToPackage{hyphens}{url}
%
\documentclass[
]{article}
\usepackage{amsmath,amssymb}
\usepackage{lmodern}
\usepackage{ifxetex,ifluatex}
\ifnum 0\ifxetex 1\fi\ifluatex 1\fi=0 % if pdftex
  \usepackage[T1]{fontenc}
  \usepackage[utf8]{inputenc}
  \usepackage{textcomp} % provide euro and other symbols
\else % if luatex or xetex
  \usepackage{unicode-math}
  \defaultfontfeatures{Scale=MatchLowercase}
  \defaultfontfeatures[\rmfamily]{Ligatures=TeX,Scale=1}
\fi
% Use upquote if available, for straight quotes in verbatim environments
\IfFileExists{upquote.sty}{\usepackage{upquote}}{}
\IfFileExists{microtype.sty}{% use microtype if available
  \usepackage[]{microtype}
  \UseMicrotypeSet[protrusion]{basicmath} % disable protrusion for tt fonts
}{}
\makeatletter
\@ifundefined{KOMAClassName}{% if non-KOMA class
  \IfFileExists{parskip.sty}{%
    \usepackage{parskip}
  }{% else
    \setlength{\parindent}{0pt}
    \setlength{\parskip}{6pt plus 2pt minus 1pt}}
}{% if KOMA class
  \KOMAoptions{parskip=half}}
\makeatother
\usepackage{xcolor}
\IfFileExists{xurl.sty}{\usepackage{xurl}}{} % add URL line breaks if available
\IfFileExists{bookmark.sty}{\usepackage{bookmark}}{\usepackage{hyperref}}
\hypersetup{
  pdftitle={Assignment 3},
  pdfauthor={Brianna Viñas},
  hidelinks,
  pdfcreator={LaTeX via pandoc}}
\urlstyle{same} % disable monospaced font for URLs
\usepackage[margin=1in]{geometry}
\usepackage{color}
\usepackage{fancyvrb}
\newcommand{\VerbBar}{|}
\newcommand{\VERB}{\Verb[commandchars=\\\{\}]}
\DefineVerbatimEnvironment{Highlighting}{Verbatim}{commandchars=\\\{\}}
% Add ',fontsize=\small' for more characters per line
\usepackage{framed}
\definecolor{shadecolor}{RGB}{248,248,248}
\newenvironment{Shaded}{\begin{snugshade}}{\end{snugshade}}
\newcommand{\AlertTok}[1]{\textcolor[rgb]{0.94,0.16,0.16}{#1}}
\newcommand{\AnnotationTok}[1]{\textcolor[rgb]{0.56,0.35,0.01}{\textbf{\textit{#1}}}}
\newcommand{\AttributeTok}[1]{\textcolor[rgb]{0.77,0.63,0.00}{#1}}
\newcommand{\BaseNTok}[1]{\textcolor[rgb]{0.00,0.00,0.81}{#1}}
\newcommand{\BuiltInTok}[1]{#1}
\newcommand{\CharTok}[1]{\textcolor[rgb]{0.31,0.60,0.02}{#1}}
\newcommand{\CommentTok}[1]{\textcolor[rgb]{0.56,0.35,0.01}{\textit{#1}}}
\newcommand{\CommentVarTok}[1]{\textcolor[rgb]{0.56,0.35,0.01}{\textbf{\textit{#1}}}}
\newcommand{\ConstantTok}[1]{\textcolor[rgb]{0.00,0.00,0.00}{#1}}
\newcommand{\ControlFlowTok}[1]{\textcolor[rgb]{0.13,0.29,0.53}{\textbf{#1}}}
\newcommand{\DataTypeTok}[1]{\textcolor[rgb]{0.13,0.29,0.53}{#1}}
\newcommand{\DecValTok}[1]{\textcolor[rgb]{0.00,0.00,0.81}{#1}}
\newcommand{\DocumentationTok}[1]{\textcolor[rgb]{0.56,0.35,0.01}{\textbf{\textit{#1}}}}
\newcommand{\ErrorTok}[1]{\textcolor[rgb]{0.64,0.00,0.00}{\textbf{#1}}}
\newcommand{\ExtensionTok}[1]{#1}
\newcommand{\FloatTok}[1]{\textcolor[rgb]{0.00,0.00,0.81}{#1}}
\newcommand{\FunctionTok}[1]{\textcolor[rgb]{0.00,0.00,0.00}{#1}}
\newcommand{\ImportTok}[1]{#1}
\newcommand{\InformationTok}[1]{\textcolor[rgb]{0.56,0.35,0.01}{\textbf{\textit{#1}}}}
\newcommand{\KeywordTok}[1]{\textcolor[rgb]{0.13,0.29,0.53}{\textbf{#1}}}
\newcommand{\NormalTok}[1]{#1}
\newcommand{\OperatorTok}[1]{\textcolor[rgb]{0.81,0.36,0.00}{\textbf{#1}}}
\newcommand{\OtherTok}[1]{\textcolor[rgb]{0.56,0.35,0.01}{#1}}
\newcommand{\PreprocessorTok}[1]{\textcolor[rgb]{0.56,0.35,0.01}{\textit{#1}}}
\newcommand{\RegionMarkerTok}[1]{#1}
\newcommand{\SpecialCharTok}[1]{\textcolor[rgb]{0.00,0.00,0.00}{#1}}
\newcommand{\SpecialStringTok}[1]{\textcolor[rgb]{0.31,0.60,0.02}{#1}}
\newcommand{\StringTok}[1]{\textcolor[rgb]{0.31,0.60,0.02}{#1}}
\newcommand{\VariableTok}[1]{\textcolor[rgb]{0.00,0.00,0.00}{#1}}
\newcommand{\VerbatimStringTok}[1]{\textcolor[rgb]{0.31,0.60,0.02}{#1}}
\newcommand{\WarningTok}[1]{\textcolor[rgb]{0.56,0.35,0.01}{\textbf{\textit{#1}}}}
\usepackage{graphicx}
\makeatletter
\def\maxwidth{\ifdim\Gin@nat@width>\linewidth\linewidth\else\Gin@nat@width\fi}
\def\maxheight{\ifdim\Gin@nat@height>\textheight\textheight\else\Gin@nat@height\fi}
\makeatother
% Scale images if necessary, so that they will not overflow the page
% margins by default, and it is still possible to overwrite the defaults
% using explicit options in \includegraphics[width, height, ...]{}
\setkeys{Gin}{width=\maxwidth,height=\maxheight,keepaspectratio}
% Set default figure placement to htbp
\makeatletter
\def\fps@figure{htbp}
\makeatother
\setlength{\emergencystretch}{3em} % prevent overfull lines
\providecommand{\tightlist}{%
  \setlength{\itemsep}{0pt}\setlength{\parskip}{0pt}}
\setcounter{secnumdepth}{-\maxdimen} % remove section numbering
\ifluatex
  \usepackage{selnolig}  % disable illegal ligatures
\fi

\title{Assignment 3}
\author{Brianna Viñas}
\date{10/16/2021}

\begin{document}
\maketitle

\hypertarget{r-markdown}{%
\subsection{R Markdown}\label{r-markdown}}

This is an R Markdown document. Markdown is a simple formatting syntax
for authoring HTML, PDF, and MS Word documents. For more details on
using R Markdown see \url{http://rmarkdown.rstudio.com}.

When you click the \textbf{Knit} button a document will be generated
that includes both content as well as the output of any embedded R code
chunks within the document. You can embed an R code chunk like this:

\begin{Shaded}
\begin{Highlighting}[]
\CommentTok{\#required libraries }
\FunctionTok{library}\NormalTok{(reshape2)}
\FunctionTok{library}\NormalTok{(gmodels)}
\FunctionTok{library}\NormalTok{(caret)}
\end{Highlighting}
\end{Shaded}

\begin{verbatim}
## Loading required package: ggplot2
\end{verbatim}

\begin{verbatim}
## Loading required package: lattice
\end{verbatim}

\begin{Shaded}
\begin{Highlighting}[]
\FunctionTok{library}\NormalTok{(ISLR)}
\FunctionTok{library}\NormalTok{(e1071)}
\end{Highlighting}
\end{Shaded}

\#Reading .CSV File and Normalization of the Data

\begin{Shaded}
\begin{Highlighting}[]
\NormalTok{UBank }\OtherTok{\textless{}{-}} \FunctionTok{read.csv}\NormalTok{(}\StringTok{"UniversalBank.csv"}\NormalTok{)}

\NormalTok{UBank}\SpecialCharTok{$}\NormalTok{Personal.Loan }\OtherTok{\textless{}{-}} \FunctionTok{factor}\NormalTok{ (UBank}\SpecialCharTok{$}\NormalTok{Personal.Loan)}
\NormalTok{UBank}\SpecialCharTok{$}\NormalTok{Online }\OtherTok{\textless{}{-}} \FunctionTok{factor}\NormalTok{(UBank}\SpecialCharTok{$}\NormalTok{Online)}
\NormalTok{UBank}\SpecialCharTok{$}\NormalTok{CreditCard}\OtherTok{\textless{}{-}}\FunctionTok{factor}\NormalTok{ (UBank}\SpecialCharTok{$}\NormalTok{CreditCard)}
\end{Highlighting}
\end{Shaded}

\#In this step, I will split the data into 60\% training and 40\%
validation.

\begin{Shaded}
\begin{Highlighting}[]
\FunctionTok{set.seed}\NormalTok{(}\DecValTok{20}\NormalTok{)}
\NormalTok{training.index}\OtherTok{\textless{}{-}}\FunctionTok{sample}\NormalTok{(}\FunctionTok{row.names}\NormalTok{(UBank), }\FloatTok{0.6}\SpecialCharTok{*}\FunctionTok{dim}\NormalTok{(UBank)[}\DecValTok{1}\NormalTok{])}
\NormalTok{valid.index}\OtherTok{\textless{}{-}} \FunctionTok{setdiff}\NormalTok{(}\FunctionTok{row.names}\NormalTok{(UBank),training.index)}
\NormalTok{train.df}\OtherTok{\textless{}{-}}\NormalTok{UBank [training.index,]}
\NormalTok{valid.df }\OtherTok{\textless{}{-}}\NormalTok{ UBank[valid.index,]}
\NormalTok{train }\OtherTok{\textless{}{-}}\NormalTok{ UBank[training.index,]}
\NormalTok{valiatingtest }\OtherTok{\textless{}{-}}\NormalTok{ UBank [training.index,]}
\end{Highlighting}
\end{Shaded}

\hypertarget{question-a-create-a-pivot-table-for-the-training-data-with-online-as-a-column-variable-cc-as-a-row-variable-and-loan-as-a-secondary-row-variable.-the-values-inside-the-table-should-convey-the-count.-in-r-use-functions-meltand-cast-or-function-table.-in-python-use-panda-dataframe-methods-meltand-pivot.}{%
\section{Question A: Create a pivot table for the training data with
Online as a column variable, CC as a row variable, and Loan as a
secondary row variable. The values inside the table should convey the
count. In R use functions melt()and cast(), or function table(). In
Python, use panda dataframe methods melt()and
pivot().}\label{question-a-create-a-pivot-table-for-the-training-data-with-online-as-a-column-variable-cc-as-a-row-variable-and-loan-as-a-secondary-row-variable.-the-values-inside-the-table-should-convey-the-count.-in-r-use-functions-meltand-cast-or-function-table.-in-python-use-panda-dataframe-methods-meltand-pivot.}}

\begin{Shaded}
\begin{Highlighting}[]
\NormalTok{melted.bank }\OtherTok{\textless{}{-}} \FunctionTok{melt}\NormalTok{(train, }\AttributeTok{id=}\FunctionTok{c}\NormalTok{(}\StringTok{"CreditCard"}\NormalTok{,}\StringTok{"Personal.Loan"}\NormalTok{), }\AttributeTok{variable=} \StringTok{"Online"}\NormalTok{)}
\end{Highlighting}
\end{Shaded}

\begin{verbatim}
## Warning: attributes are not identical across measure variables; they will be
## dropped
\end{verbatim}

\begin{Shaded}
\begin{Highlighting}[]
\NormalTok{cast.bank }\OtherTok{\textless{}{-}}\FunctionTok{dcast}\NormalTok{(melted.bank,CreditCard}\SpecialCharTok{+}\NormalTok{Personal.Loan}\SpecialCharTok{\textasciitilde{}}\NormalTok{Online)}
\end{Highlighting}
\end{Shaded}

\begin{verbatim}
## Aggregation function missing: defaulting to length
\end{verbatim}

\begin{Shaded}
\begin{Highlighting}[]
\NormalTok{cast.bank [,}\FunctionTok{c}\NormalTok{(}\DecValTok{1}\SpecialCharTok{:}\DecValTok{2}\NormalTok{,}\DecValTok{14}\NormalTok{)]}
\end{Highlighting}
\end{Shaded}

\begin{verbatim}
##   CreditCard Personal.Loan Online
## 1          0             0   1878
## 2          0             1    212
## 3          1             0    822
## 4          1             1     88
\end{verbatim}

\hypertarget{question-b-consider-the-task-of-classifying-a-customer-who-owns-a-bank-credit-card-and-is-actively-using-online-banking-services.-looking-at-the-pivot-table-what-is-the-probability-that-this-customer-will-accept-the-loan-offer-this-is-the-probability-of-loan-acceptance-loan-1-conditional-on-having-a-bank-credit-card-cc-1-and-being-an-active-user-of-online-banking-services-online-1}{%
\section{Question B: Consider the task of classifying a customer who
owns a bank credit card and is actively using online banking services.
Looking at the pivot table, what is the probability that this customer
will accept the loan offer? {[}This is the probability of loan
acceptance (Loan = 1) conditional on having a bank credit card (CC = 1)
and being an active user of online banking services (Online =
1){]}}\label{question-b-consider-the-task-of-classifying-a-customer-who-owns-a-bank-credit-card-and-is-actively-using-online-banking-services.-looking-at-the-pivot-table-what-is-the-probability-that-this-customer-will-accept-the-loan-offer-this-is-the-probability-of-loan-acceptance-loan-1-conditional-on-having-a-bank-credit-card-cc-1-and-being-an-active-user-of-online-banking-services-online-1}}

\begin{Shaded}
\begin{Highlighting}[]
\NormalTok{a}\OtherTok{=}\FunctionTok{table}\NormalTok{(train[,}\FunctionTok{c}\NormalTok{(}\DecValTok{10}\NormalTok{,}\DecValTok{13}\NormalTok{,}\DecValTok{14}\NormalTok{)])}
\NormalTok{b}\OtherTok{\textless{}{-}}\FunctionTok{as.data.frame}\NormalTok{(a)}
\NormalTok{b}
\end{Highlighting}
\end{Shaded}

\begin{verbatim}
##   Personal.Loan Online CreditCard Freq
## 1             0      0          0  769
## 2             1      0          0   80
## 3             0      1          0 1109
## 4             1      1          0  132
## 5             0      0          1  309
## 6             1      0          1   41
## 7             0      1          1  513
## 8             1      1          1   47
\end{verbatim}

\hypertarget{question-c-create-two-separate-pivot-tables-for-the-training-data.-onewill-have-loan-rows-as-a-function-of-online-columns-and-the-other-will-have-loan-rows-as-a-function-of-cc.}{%
\section{Question C: Create two separate pivot tables for the training
data. Onewill have Loan (rows) as a function of Online (columns) and the
other will have Loan (rows) as a function of
CC.}\label{question-c-create-two-separate-pivot-tables-for-the-training-data.-onewill-have-loan-rows-as-a-function-of-online-columns-and-the-other-will-have-loan-rows-as-a-function-of-cc.}}

\begin{Shaded}
\begin{Highlighting}[]
\CommentTok{\# I will also be creating a pivot table for Loan (rows) as a function of Online (columns).}
\FunctionTok{table}\NormalTok{(train[,}\FunctionTok{c}\NormalTok{(}\DecValTok{10}\NormalTok{,}\DecValTok{13}\NormalTok{)])}
\end{Highlighting}
\end{Shaded}

\begin{verbatim}
##              Online
## Personal.Loan    0    1
##             0 1078 1622
##             1  121  179
\end{verbatim}

\begin{Shaded}
\begin{Highlighting}[]
\CommentTok{\#This is the pivot table for Loan (rows) as a function of cc. }
\FunctionTok{table}\NormalTok{ (train[,}\FunctionTok{c}\NormalTok{(}\DecValTok{10}\NormalTok{,}\DecValTok{14}\NormalTok{)])}
\end{Highlighting}
\end{Shaded}

\begin{verbatim}
##              CreditCard
## Personal.Loan    0    1
##             0 1878  822
##             1  212   88
\end{verbatim}

\hypertarget{question-d-part-1-compute-the-following-quantities-pa-b-means-the-probability-ofa-given-b-i.pcc-1-loan-1-the-proportion-of-credit-card-holders-among-the-loan-acceptors}{%
\section{Question D, Part 1: Compute the following quantities {[}P(A
\textbar{} B) means ``the probability ofA given B''{]}: i.P(CC = 1
\textbar{} Loan = 1) (the proportion of credit card holders among the
loan
acceptors)}\label{question-d-part-1-compute-the-following-quantities-pa-b-means-the-probability-ofa-given-b-i.pcc-1-loan-1-the-proportion-of-credit-card-holders-among-the-loan-acceptors}}

\begin{Shaded}
\begin{Highlighting}[]
\NormalTok{P1}\OtherTok{\textless{}{-}}\FunctionTok{table}\NormalTok{(train[,}\FunctionTok{c}\NormalTok{(}\DecValTok{14}\NormalTok{,}\DecValTok{10}\NormalTok{)])}
\NormalTok{P1}
\end{Highlighting}
\end{Shaded}

\begin{verbatim}
##           Personal.Loan
## CreditCard    0    1
##          0 1878  212
##          1  822   88
\end{verbatim}

\begin{Shaded}
\begin{Highlighting}[]
\CommentTok{\#Answer: 88/(88+212) = 0.29333333}
\end{Highlighting}
\end{Shaded}

\#Question D, Part 2: .P(Online = 1 \textbar{} Loan = 1)

\begin{Shaded}
\begin{Highlighting}[]
\NormalTok{P2}\OtherTok{\textless{}{-}}\FunctionTok{table}\NormalTok{(train[,}\FunctionTok{c}\NormalTok{(}\DecValTok{13}\NormalTok{,}\DecValTok{10}\NormalTok{)])}
\NormalTok{P2}
\end{Highlighting}
\end{Shaded}

\begin{verbatim}
##       Personal.Loan
## Online    0    1
##      0 1078  121
##      1 1622  179
\end{verbatim}

\begin{Shaded}
\begin{Highlighting}[]
\CommentTok{\#Answer: 179/(179+121)= 0.59666667}
\end{Highlighting}
\end{Shaded}

\#Question D, Part 3: P(Loan = 1) (the proportion of loan acceptors)

\begin{Shaded}
\begin{Highlighting}[]
\NormalTok{P3}\OtherTok{\textless{}{-}}\FunctionTok{table}\NormalTok{(train[,}\DecValTok{10}\NormalTok{])}
\NormalTok{P3}
\end{Highlighting}
\end{Shaded}

\begin{verbatim}
## 
##    0    1 
## 2700  300
\end{verbatim}

\begin{Shaded}
\begin{Highlighting}[]
\CommentTok{\#Answer:300/(300+2700)= 0.1}
\end{Highlighting}
\end{Shaded}

\#Question D, Part 4: P(CC = 1 \textbar{} Loan = 0)

\begin{Shaded}
\begin{Highlighting}[]
\NormalTok{P4 }\OtherTok{\textless{}{-}}\FunctionTok{table}\NormalTok{(train[,}\FunctionTok{c}\NormalTok{(}\DecValTok{14}\NormalTok{,}\DecValTok{10}\NormalTok{)])}
\NormalTok{P4}
\end{Highlighting}
\end{Shaded}

\begin{verbatim}
##           Personal.Loan
## CreditCard    0    1
##          0 1878  212
##          1  822   88
\end{verbatim}

\begin{Shaded}
\begin{Highlighting}[]
\CommentTok{\#Answer: 822/(822+1878)= 0.30444444}
\end{Highlighting}
\end{Shaded}

\#Question D, Part 5 P(Online = 1 \textbar{} Loan = 0)

\begin{Shaded}
\begin{Highlighting}[]
\NormalTok{P5}\OtherTok{\textless{}{-}}\FunctionTok{table}\NormalTok{(train[,}\FunctionTok{c}\NormalTok{(}\DecValTok{13}\NormalTok{,}\DecValTok{10}\NormalTok{)])}
\NormalTok{P5}
\end{Highlighting}
\end{Shaded}

\begin{verbatim}
##       Personal.Loan
## Online    0    1
##      0 1078  121
##      1 1622  179
\end{verbatim}

\begin{Shaded}
\begin{Highlighting}[]
\CommentTok{\#Answer: 1622/(1622+1078)= 0.60074074}
\end{Highlighting}
\end{Shaded}

\#Question D, Part 6 P(Loan = 0)

\begin{Shaded}
\begin{Highlighting}[]
\NormalTok{P6}\OtherTok{\textless{}{-}}\FunctionTok{table}\NormalTok{(train[,}\DecValTok{10}\NormalTok{])}
\NormalTok{P6}
\end{Highlighting}
\end{Shaded}

\begin{verbatim}
## 
##    0    1 
## 2700  300
\end{verbatim}

\begin{Shaded}
\begin{Highlighting}[]
\CommentTok{\#Answer: 2700/(2700+300)= 0.9}
\end{Highlighting}
\end{Shaded}

\hypertarget{question-e-use-the-quantities-computed-above-to-compute-the-naive-bayes-probability-ploan-1-cc-1-online-1.}{%
\section{Question E: Use the quantities computed above to compute the
naive Bayes probability P(Loan = 1 \textbar{} CC = 1, Online =
1).}\label{question-e-use-the-quantities-computed-above-to-compute-the-naive-bayes-probability-ploan-1-cc-1-online-1.}}

\#Answer: \#Naive Bayes Probability=
(P1\emph{P2}P3)/{[}(P1\emph{P2}P3)+(P4\emph{P5}P6){]} \#(0.29333333)*
(0.59666667)\emph{(0.1)/(0.30444444)}(0.60074074)*(0.9)
\#0.1750222/(0.01750222+0.16460296) \#0.0175022/0.18210518 \# Answer =
0.0961105

\#Question F: Compare this value with the one obtained from the pivot
table in (B). Which is a more accurate estimate? \#Answer: \#The value
obtained fromt the pivot table is 0.0942285 and the value obtained from
the naive bayes is O.0961105. These values are almost equivalent to one
another. However, the pivot table is a more accurate value. This is
because each of the conditional probabilities are manually caluclated
one by one.

\#Question G: Which of the entries in this table are needed for
computing P(Loan = 1 \textbar{} CC = 1, Online = 1)? Run naive Bayes on
the data. Examine the model output on training data, and find the entry
that corresponds to P(Loan = 1 \textbar{} CC = 1, Online = 1). Compare
this to the number you obtained in (E). \#In this step, I will perform
Naive Bayes on training data

\begin{Shaded}
\begin{Highlighting}[]
\FunctionTok{table}\NormalTok{(train[,}\FunctionTok{c}\NormalTok{(}\DecValTok{10}\NormalTok{,}\DecValTok{13}\SpecialCharTok{:}\DecValTok{14}\NormalTok{)])}
\end{Highlighting}
\end{Shaded}

\begin{verbatim}
## , , CreditCard = 0
## 
##              Online
## Personal.Loan    0    1
##             0  769 1109
##             1   80  132
## 
## , , CreditCard = 1
## 
##              Online
## Personal.Loan    0    1
##             0  309  513
##             1   41   47
\end{verbatim}

\begin{Shaded}
\begin{Highlighting}[]
\NormalTok{training\_Naive}\OtherTok{\textless{}{-}}\NormalTok{train[,}\FunctionTok{c}\NormalTok{(}\DecValTok{10}\NormalTok{,}\DecValTok{13}\SpecialCharTok{:}\DecValTok{14}\NormalTok{)]}
\NormalTok{UBank\_NB}\OtherTok{\textless{}{-}}\FunctionTok{naiveBayes}\NormalTok{(Personal.Loan}\SpecialCharTok{\textasciitilde{}}\NormalTok{.,}\AttributeTok{data=}\NormalTok{ training\_Naive) }
\NormalTok{UBank\_NB}
\end{Highlighting}
\end{Shaded}

\begin{verbatim}
## 
## Naive Bayes Classifier for Discrete Predictors
## 
## Call:
## naiveBayes.default(x = X, y = Y, laplace = laplace)
## 
## A-priori probabilities:
## Y
##   0   1 
## 0.9 0.1 
## 
## Conditional probabilities:
##    Online
## Y           0         1
##   0 0.3992593 0.6007407
##   1 0.4033333 0.5966667
## 
##    CreditCard
## Y           0         1
##   0 0.6955556 0.3044444
##   1 0.7066667 0.2933333
\end{verbatim}

\#The value which is obtained is 0.0961105 after running Naive Bayes on
data were the value calculated from E is 0.0961105. The two values are
exactly the same.

\end{document}
