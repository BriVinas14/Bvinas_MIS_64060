% Options for packages loaded elsewhere
\PassOptionsToPackage{unicode}{hyperref}
\PassOptionsToPackage{hyphens}{url}
%
\documentclass[
]{article}
\usepackage{amsmath,amssymb}
\usepackage{lmodern}
\usepackage{ifxetex,ifluatex}
\ifnum 0\ifxetex 1\fi\ifluatex 1\fi=0 % if pdftex
  \usepackage[T1]{fontenc}
  \usepackage[utf8]{inputenc}
  \usepackage{textcomp} % provide euro and other symbols
\else % if luatex or xetex
  \usepackage{unicode-math}
  \defaultfontfeatures{Scale=MatchLowercase}
  \defaultfontfeatures[\rmfamily]{Ligatures=TeX,Scale=1}
\fi
% Use upquote if available, for straight quotes in verbatim environments
\IfFileExists{upquote.sty}{\usepackage{upquote}}{}
\IfFileExists{microtype.sty}{% use microtype if available
  \usepackage[]{microtype}
  \UseMicrotypeSet[protrusion]{basicmath} % disable protrusion for tt fonts
}{}
\makeatletter
\@ifundefined{KOMAClassName}{% if non-KOMA class
  \IfFileExists{parskip.sty}{%
    \usepackage{parskip}
  }{% else
    \setlength{\parindent}{0pt}
    \setlength{\parskip}{6pt plus 2pt minus 1pt}}
}{% if KOMA class
  \KOMAoptions{parskip=half}}
\makeatother
\usepackage{xcolor}
\IfFileExists{xurl.sty}{\usepackage{xurl}}{} % add URL line breaks if available
\IfFileExists{bookmark.sty}{\usepackage{bookmark}}{\usepackage{hyperref}}
\hypersetup{
  pdftitle={Final Project\_Customer Segmentation},
  pdfauthor={Brianna Viñas},
  hidelinks,
  pdfcreator={LaTeX via pandoc}}
\urlstyle{same} % disable monospaced font for URLs
\usepackage[margin=1in]{geometry}
\usepackage{color}
\usepackage{fancyvrb}
\newcommand{\VerbBar}{|}
\newcommand{\VERB}{\Verb[commandchars=\\\{\}]}
\DefineVerbatimEnvironment{Highlighting}{Verbatim}{commandchars=\\\{\}}
% Add ',fontsize=\small' for more characters per line
\usepackage{framed}
\definecolor{shadecolor}{RGB}{248,248,248}
\newenvironment{Shaded}{\begin{snugshade}}{\end{snugshade}}
\newcommand{\AlertTok}[1]{\textcolor[rgb]{0.94,0.16,0.16}{#1}}
\newcommand{\AnnotationTok}[1]{\textcolor[rgb]{0.56,0.35,0.01}{\textbf{\textit{#1}}}}
\newcommand{\AttributeTok}[1]{\textcolor[rgb]{0.77,0.63,0.00}{#1}}
\newcommand{\BaseNTok}[1]{\textcolor[rgb]{0.00,0.00,0.81}{#1}}
\newcommand{\BuiltInTok}[1]{#1}
\newcommand{\CharTok}[1]{\textcolor[rgb]{0.31,0.60,0.02}{#1}}
\newcommand{\CommentTok}[1]{\textcolor[rgb]{0.56,0.35,0.01}{\textit{#1}}}
\newcommand{\CommentVarTok}[1]{\textcolor[rgb]{0.56,0.35,0.01}{\textbf{\textit{#1}}}}
\newcommand{\ConstantTok}[1]{\textcolor[rgb]{0.00,0.00,0.00}{#1}}
\newcommand{\ControlFlowTok}[1]{\textcolor[rgb]{0.13,0.29,0.53}{\textbf{#1}}}
\newcommand{\DataTypeTok}[1]{\textcolor[rgb]{0.13,0.29,0.53}{#1}}
\newcommand{\DecValTok}[1]{\textcolor[rgb]{0.00,0.00,0.81}{#1}}
\newcommand{\DocumentationTok}[1]{\textcolor[rgb]{0.56,0.35,0.01}{\textbf{\textit{#1}}}}
\newcommand{\ErrorTok}[1]{\textcolor[rgb]{0.64,0.00,0.00}{\textbf{#1}}}
\newcommand{\ExtensionTok}[1]{#1}
\newcommand{\FloatTok}[1]{\textcolor[rgb]{0.00,0.00,0.81}{#1}}
\newcommand{\FunctionTok}[1]{\textcolor[rgb]{0.00,0.00,0.00}{#1}}
\newcommand{\ImportTok}[1]{#1}
\newcommand{\InformationTok}[1]{\textcolor[rgb]{0.56,0.35,0.01}{\textbf{\textit{#1}}}}
\newcommand{\KeywordTok}[1]{\textcolor[rgb]{0.13,0.29,0.53}{\textbf{#1}}}
\newcommand{\NormalTok}[1]{#1}
\newcommand{\OperatorTok}[1]{\textcolor[rgb]{0.81,0.36,0.00}{\textbf{#1}}}
\newcommand{\OtherTok}[1]{\textcolor[rgb]{0.56,0.35,0.01}{#1}}
\newcommand{\PreprocessorTok}[1]{\textcolor[rgb]{0.56,0.35,0.01}{\textit{#1}}}
\newcommand{\RegionMarkerTok}[1]{#1}
\newcommand{\SpecialCharTok}[1]{\textcolor[rgb]{0.00,0.00,0.00}{#1}}
\newcommand{\SpecialStringTok}[1]{\textcolor[rgb]{0.31,0.60,0.02}{#1}}
\newcommand{\StringTok}[1]{\textcolor[rgb]{0.31,0.60,0.02}{#1}}
\newcommand{\VariableTok}[1]{\textcolor[rgb]{0.00,0.00,0.00}{#1}}
\newcommand{\VerbatimStringTok}[1]{\textcolor[rgb]{0.31,0.60,0.02}{#1}}
\newcommand{\WarningTok}[1]{\textcolor[rgb]{0.56,0.35,0.01}{\textbf{\textit{#1}}}}
\usepackage{graphicx}
\makeatletter
\def\maxwidth{\ifdim\Gin@nat@width>\linewidth\linewidth\else\Gin@nat@width\fi}
\def\maxheight{\ifdim\Gin@nat@height>\textheight\textheight\else\Gin@nat@height\fi}
\makeatother
% Scale images if necessary, so that they will not overflow the page
% margins by default, and it is still possible to overwrite the defaults
% using explicit options in \includegraphics[width, height, ...]{}
\setkeys{Gin}{width=\maxwidth,height=\maxheight,keepaspectratio}
% Set default figure placement to htbp
\makeatletter
\def\fps@figure{htbp}
\makeatother
\setlength{\emergencystretch}{3em} % prevent overfull lines
\providecommand{\tightlist}{%
  \setlength{\itemsep}{0pt}\setlength{\parskip}{0pt}}
\setcounter{secnumdepth}{-\maxdimen} % remove section numbering
\ifluatex
  \usepackage{selnolig}  % disable illegal ligatures
\fi

\title{Final Project\_Customer Segmentation}
\author{Brianna Viñas}
\date{12/5/2021}

\begin{document}
\maketitle

\begin{Shaded}
\begin{Highlighting}[]
\CommentTok{\#First, I will read the .csv file. }
\FunctionTok{library}\NormalTok{(factoextra)}
\end{Highlighting}
\end{Shaded}

\begin{verbatim}
## Loading required package: ggplot2
\end{verbatim}

\begin{verbatim}
## Welcome! Want to learn more? See two factoextra-related books at https://goo.gl/ve3WBa
\end{verbatim}

\begin{Shaded}
\begin{Highlighting}[]
\NormalTok{cust}\OtherTok{\textless{}{-}} \FunctionTok{read.csv}\NormalTok{(}\StringTok{"Customers.csv"}\NormalTok{)}
\FunctionTok{head}\NormalTok{(cust)}
\end{Highlighting}
\end{Shaded}

\begin{verbatim}
##   CustomerID Gender Age Annual.Income..k.. Spending.Score..1.100.
## 1          1   Male  19                 15                     39
## 2          2   Male  21                 15                     81
## 3          3 Female  20                 16                      6
## 4          4 Female  23                 16                     77
## 5          5 Female  31                 17                     40
## 6          6 Female  22                 17                     76
\end{verbatim}

\begin{Shaded}
\begin{Highlighting}[]
\CommentTok{\#Taking the quantitative variables in order to scale. }
\NormalTok{cust1}\OtherTok{\textless{}{-}}\NormalTok{cust[,}\DecValTok{4}\SpecialCharTok{:}\DecValTok{5}\NormalTok{]}
\FunctionTok{head}\NormalTok{(cust1)}
\end{Highlighting}
\end{Shaded}

\begin{verbatim}
##   Annual.Income..k.. Spending.Score..1.100.
## 1                 15                     39
## 2                 15                     81
## 3                 16                      6
## 4                 16                     77
## 5                 17                     40
## 6                 17                     76
\end{verbatim}

\begin{Shaded}
\begin{Highlighting}[]
\CommentTok{\#Finding the value of K{-}means using unsupervised learning. Wanted to use the simplest, but most accurate method possible. }
\FunctionTok{fviz\_nbclust}\NormalTok{(cust1,kmeans,}\AttributeTok{method=}\StringTok{"wss"}\NormalTok{)}\SpecialCharTok{+}\FunctionTok{geom\_vline}\NormalTok{(}\AttributeTok{xintercept =} \DecValTok{2}\NormalTok{,}\AttributeTok{linetype=} \DecValTok{5}\NormalTok{)}\SpecialCharTok{+}\FunctionTok{labs}\NormalTok{(}\AttributeTok{subtitle =} \StringTok{"Elbow Method"}\NormalTok{)}
\end{Highlighting}
\end{Shaded}

\includegraphics{Final-Project_files/figure-latex/unnamed-chunk-3-1.pdf}

\begin{Shaded}
\begin{Highlighting}[]
\FunctionTok{fviz\_nbclust}\NormalTok{(cust1,kmeans,}\AttributeTok{method =}\StringTok{"silhouette"}\NormalTok{) }\SpecialCharTok{+} \FunctionTok{labs}\NormalTok{ (}\AttributeTok{subtitle =} \StringTok{"Silhouette Method"}\NormalTok{)}
\end{Highlighting}
\end{Shaded}

\includegraphics{Final-Project_files/figure-latex/unnamed-chunk-3-2.pdf}

\begin{Shaded}
\begin{Highlighting}[]
\CommentTok{\#Here, I will set the seed for kmeans. }
\FunctionTok{set.seed}\NormalTok{(}\DecValTok{1}\NormalTok{)}
\NormalTok{k5}\OtherTok{\textless{}{-}}\FunctionTok{kmeans}\NormalTok{(cust1, }\AttributeTok{centers =} \DecValTok{2}\NormalTok{, }\AttributeTok{nstart =} \DecValTok{25}\NormalTok{)}
\NormalTok{k5}\SpecialCharTok{$}\NormalTok{centers}
\end{Highlighting}
\end{Shaded}

\begin{verbatim}
##   Annual.Income..k.. Spending.Score..1.100.
## 1           79.60000               50.12727
## 2           37.28889               50.28889
\end{verbatim}

\begin{Shaded}
\begin{Highlighting}[]
\CommentTok{\#Thus, K= 5, meaning that there will be 5 clusters. }
\end{Highlighting}
\end{Shaded}

\begin{Shaded}
\begin{Highlighting}[]
\CommentTok{\#Clustering the data from .csv file. }
\NormalTok{custclus}\OtherTok{\textless{}{-}}\FunctionTok{kmeans}\NormalTok{(cust1,}\DecValTok{5}\NormalTok{)}
\NormalTok{custclus}
\end{Highlighting}
\end{Shaded}

\begin{verbatim}
## K-means clustering with 5 clusters of sizes 81, 35, 22, 39, 23
## 
## Cluster means:
##   Annual.Income..k.. Spending.Score..1.100.
## 1           55.29630               49.51852
## 2           88.20000               17.11429
## 3           25.72727               79.36364
## 4           86.53846               82.12821
## 5           26.30435               20.91304
## 
## Clustering vector:
##   [1] 5 3 5 3 5 3 5 3 5 3 5 3 5 3 5 3 5 3 5 3 5 3 5 3 5 3 5 3 5 3 5 3 5 3 5 3 5
##  [38] 3 5 3 5 3 5 1 5 3 1 1 1 1 1 1 1 1 1 1 1 1 1 1 1 1 1 1 1 1 1 1 1 1 1 1 1 1
##  [75] 1 1 1 1 1 1 1 1 1 1 1 1 1 1 1 1 1 1 1 1 1 1 1 1 1 1 1 1 1 1 1 1 1 1 1 1 1
## [112] 1 1 1 1 1 1 1 1 1 1 1 1 4 2 4 1 4 2 4 2 4 1 4 2 4 2 4 2 4 2 4 1 4 2 4 2 4
## [149] 2 4 2 4 2 4 2 4 2 4 2 4 2 4 2 4 2 4 2 4 2 4 2 4 2 4 2 4 2 4 2 4 2 4 2 4 2
## [186] 4 2 4 2 4 2 4 2 4 2 4 2 4 2 4
## 
## Within cluster sum of squares by cluster:
## [1]  9875.111 12511.143  3519.455 13444.051  5098.696
##  (between_SS / total_SS =  83.5 %)
## 
## Available components:
## 
## [1] "cluster"      "centers"      "totss"        "withinss"     "tot.withinss"
## [6] "betweenss"    "size"         "iter"         "ifault"
\end{verbatim}

\begin{Shaded}
\begin{Highlighting}[]
\CommentTok{\#Now, I will visualize the clusters. }
\FunctionTok{ggplot}\NormalTok{(cust1, }\FunctionTok{aes}\NormalTok{(}\AttributeTok{x =}\NormalTok{ Annual.Income..k..,}\AttributeTok{y =}\NormalTok{ Spending.Score..}\DecValTok{1}\NormalTok{.}\FloatTok{100.}\NormalTok{)) }\SpecialCharTok{+}\FunctionTok{geom\_point}\NormalTok{(}\AttributeTok{stat =} \StringTok{"identity"}\NormalTok{, }\FunctionTok{aes}\NormalTok{(}\AttributeTok{color=}\FunctionTok{as.factor}\NormalTok{(custclus}\SpecialCharTok{$}\NormalTok{cluster)))}\SpecialCharTok{+} \FunctionTok{scale\_color\_discrete}\NormalTok{(}\AttributeTok{name=}\StringTok{"k"}\NormalTok{,}\AttributeTok{breaks=}\FunctionTok{c}\NormalTok{(}\StringTok{"1"}\NormalTok{, }\StringTok{"2"}\NormalTok{, }\StringTok{"3"}\NormalTok{, }\StringTok{"4"}\NormalTok{, }\StringTok{"5"}\NormalTok{),}\AttributeTok{labels=}\FunctionTok{c}\NormalTok{(}\StringTok{"Cluster1"}\NormalTok{, }\StringTok{"Cluster2"}\NormalTok{,}\StringTok{"Cluster3"}\NormalTok{, }\StringTok{"Cluster4"}\NormalTok{, }\StringTok{"Cluster5"}\NormalTok{))}\SpecialCharTok{+} \FunctionTok{ggtitle}\NormalTok{(}\StringTok{"Customer Segmentation"}\NormalTok{)}
\end{Highlighting}
\end{Shaded}

\includegraphics{Final-Project_files/figure-latex/unnamed-chunk-4-1.pdf}

\begin{Shaded}
\begin{Highlighting}[]
\CommentTok{\#Thus, the following can be concluded:}
\CommentTok{\#Cluster 1 are customers who earn a medium annual income and have a medium annual spending rate. }
\CommentTok{\#Cluster 2 customers who have a high annual income and a low annual spending rate. }
\CommentTok{\#Cluster 3 costumers who have low annual incomes and a high annual spending rates.}
\CommentTok{\#Cluster 4 customers with high annual incomes and have high annual spending rates. }
\CommentTok{\#Cluster 5 shows that customers with low annual incomes and low annual spending rates. }
\end{Highlighting}
\end{Shaded}


\end{document}
